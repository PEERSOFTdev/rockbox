% $Id$ %
\chapter{Browsing and playing}
\section{\label{ref:file_browser}File Browser}
\screenshot{rockbox_interface/images/ss-file-browser}{The file browser}{}
Rockbox lets you browse your music in either of two ways. The 
\setting{File Browser} lets you navigate through the files and directories on 
your \dap, entering directories and executing the default action on each file.
To help differentiate files, each file format is displayed with an icon. 

The \setting{Database Browser}, on the other hand, allows you to navigate 
through the music on your player using categories like album, artist, genre,
etc.

You can select whether to browse using the \setting{File Browser} or the
\setting{Database Browser} by selecting either \setting{Files} or
\setting{Database} in the \setting{Main Menu}.
If you choose the \setting{File Browser}, the \setting{Show Files} setting
lets you select what types of files you wish to view. See
\reference{ref:ShowFiles} for more information on the \setting{Show Files}
setting.

\note{The \setting{File Browser} allows you to manipulate your files in ways
that are not available within the \setting{Database Browser}. Read more about
\setting{Database} in \reference{ref:database}. The remainder of this section
deals with the \setting{File Browser}.}

\opt{ondio}{
Unlike the Archos Firmware, Rockbox provides multivolume support for the
MultiMediaCard, this means the \dap{} can access both data volumes (internal
memory and the MMC), thus being able to for instance, build playlists with
files from both volumes.
In the \setting{File Browser} a new directory will appear as soon as the device
has read the content after inserting the card. This new directory's name is
generated as \fname{<MMC1>}, and will behave exactly as any other directory
on the \dap{}.
}

\opt{iriverh10,iriverh10_5gb}{\note{
If your \dap{} is a MTP model, the Music directory where all your music is stored
may be hidden in the \setting{File Browser}. This may be fixed by either
changing its properties (on a computer) to not hidden, or by changing
the \setting{Show Files} setting to all.
}}

\subsection{\label{ref:controls}File Browser Controls}
\begin{btnmap}
      \ActionStdPrev{}/\ActionStdNext{}
      \opt{HAVEREMOTEKEYMAP}{& \ActionRCStdPrev{}/\ActionRCStdNext{}}
         & Go to previous/next item in list. If you are on the first/last 
           entry, the cursor will wrap to the last/first entry.\\
      %
      \opt{IRIVER_H100_PAD,IRIVER_H300_PAD,RECORDER_PAD}
        {
          \ButtonOn+\ButtonUp{}/ \ButtonDown
          \opt{HAVEREMOTEKEYMAP}{&
            \opt{IRIVER_RC_H100_PAD}{\ButtonRCSource{}/ \ButtonRCBitrate}
          }
          & Move one page up/down in the list.\\
        }
      \opt{IRIVER_H10_PAD}
        {
          \ButtonRew{}/ \ButtonFF
          & Move one page up/down in the list.\\
        }
      %
      \ActionTreeParentDirectory
      \opt{HAVEREMOTEKEYMAP}{& \ActionRCTreeParentDirectory}
      & Go to the parent directory.\\
      %
      \ActionTreeEnter
      \opt{HAVEREMOTEKEYMAP}{& \ActionRCTreeEnter}
      & Execute the default action on the selected file or enter a
        directory.\\
      %
      \ActionTreeWps 
      \opt{HAVEREMOTEKEYMAP}{& \ActionRCTreeWps}
         & If there is an audio file playing, return to the
         \setting{While Playing Screen} (WPS) without stopping playback.\\
      %
      \nopt{player,SANSA_C200_PAD}%
        {%
          \ActionTreeStop 
          \opt{HAVEREMOTEKEYMAP}{& \ActionRCTreeStop}
          & Stop audio playback.\\%
        }%
      %
      \ActionStdContext{}
      \opt{HAVEREMOTEKEYMAP}{& \ActionRCStdContext}
      & Enter the \setting{Context Menu}.\\
      %
      \ActionStdMenu{}
      \opt{HAVEREMOTEKEYMAP}{& \ActionRCStdMenu}
      & Enter the \setting{Main Menu}.\\
      %
      \opt{quickscreen}{
        \ActionStdQuickScreen
        \opt{HAVEREMOTEKEYMAP}{& \ActionRCStdQuickScreen}
        & Switch to the \setting{Quick Screen}
        (see \reference{ref:QuickScreen}). \\
      }
      \opt{RECORDER_PAD}{
        \ButtonFThree & Switch to the \setting{Quick Screen}.\\ 
        %
      }
      %
      \opt{SANSA_E200_PAD}{
        \ActionStdRec & Switch to the \setting{Recording Screen}.\\
      %
      }
      \nopt{touchscreen}{\opt{hotkey}{
        \ActionTreeHotkey
            &
        \opt{HAVEREMOTEKEYMAP}{
            &}
        Activate the \setting{Hotkey} function
        (see \reference{ref:Hotkeys}).
            \\
      }}
\end{btnmap}

\opt{RECORDER_PAD}{
  The functions of the F keys are also summarised on the button bar at the
  bottom of the screen.
}

\subsection{\label{ref:Contextmenu}\label{ref:PartIISectionFM}Context Menu}
\screenshot{rockbox_interface/images/ss-context-menu}{The Context Menu}{}

The \setting{Context Menu} allows you to perform certain operations on files or 
directories.  To access the \setting{Context Menu}, position the selector over a file 
or directory and access the context menu with \ActionStdContext{}.\\

\note{The \setting{Context Menu} is a context sensitive menu.  If the 
\setting{Context Menu} is invoked on a file, it will display options available 
for files.  If the \setting{Context Menu} is invoked on a directory, 
it will display options for directories.\\}

The \setting{Context Menu} contains the following options (unless otherwise noted, 
each option pertains both to files and directories):

\begin{description}
\item [Playlist.]
  Enters the \setting{Playlist Submenu} (see \reference{ref:playlist_submenu}).
\item [Playlist Catalogue.]
  Enters the \setting{Playlist Catalogue Submenu} (see 
  \reference{ref:playlist_catalogue}).
\item [Rename.]
  This function lets the user modify the name of a file or directory.
\item [Cut.]
  Copies the name of the currently selected file or directory to the clipboard
  and marks it to be `cut'.
\item [Copy.]
  Copies the name of the currently selected file or directory to the clipboard
  and marks it to be `copied'.
\item [Paste.]
  Only visible if a file or directory name is on the clipboard. When selected
  it will move or copy the clipboard to the current directory.
\item [Delete.]
  Deletes the currently selected file. This option applies only to files, and
  not to directories. Rockbox will ask for confirmation before deleting a file.
  Press \ActionYesNoAccept{}
  to confirm deletion or any other key to cancel.
\item [Delete Directory.]
  Deletes the currently selected directory and all of the files and subdirectories
  it may contain. Deleted directories cannot be recovered. Use this feature with
  caution!
\opt{lcd_non-mono}{
\item [Set As Backdrop.]
  Set the selected \fname{bmp} file as background image. The bitmaps need to meet the
  conditions explained in \reference{ref:LoadingBackdrops}.
}
\item [Open with.]
  Runs a viewer plugin on the file. Normally, when a file is selected in Rockbox,
  Rockbox automatically detects the file type and runs the appropriate plugin.
  The \setting{Open With} function can be used to override the default action and
  select a viewer by hand.
  For example, this function can be used to view a text file
  even if the file has a non-standard extension (i.e., the file has an extension
  of something other than \fname{.txt}). See \reference{ref:Viewersplugins}
  for more details on viewers.
\item [Create Directory.]
  Create a new directory in the current directory on the disk.
\item [Properties.]
  Shows properties such as size and the time and date of the last modification
  for the selected file. If used on a directory, the number of files and
  subdirectories will be shown, as well as the total size.
\opt{recording}{
  \item [Set As Recording Directory.]
    Save recordings in the selected directory.
}
\item [\label{ref:StartFileBrowserHere}Start File Browser Here.]
  This option allows users to set the currently selected directory as the default
  start directory for the file browser. This option is not available for files.
  \note{If you have \setting{Auto-Change Directory} and
  \setting{Constrain Auto-Change} enabled, the directories returned will
  be constrained to the directory you have chosen here and those below it.
  See \reference{ref:ConstrainAutoChange}}
\item [Add to Shortcuts.]
  Adds a link to the selected item in the \fname{/.rockbox/shortcuts.txt} file.
  If the file does not already exist it will be created.
\end{description}

\subsection{\label{sec:virtual_keyboard}Virtual Keyboard}
\screenshot{rockbox_interface/images/ss-virtual-keyboard}{The virtual keyboard}{}
This is the virtual keyboard that is used when entering text in Rockbox, for 
example when renaming a file or creating a new directory.
\nopt{player}{The virtual keyboard can be easily changed by making a text file
 with the required layout. More information on how to achieve this can be found
 on the Rockbox website at \wikilink{LoadableKeyboardLayouts}.}

\opt{morse_input}{
  Also you can switch to Morse code input mode by changing the
  \setting{Use Morse Code Input} setting%
  \opt{IRIVER_H100_PAD,IRIVER_H300_PAD,IPOD_4G_PAD,IPOD_3G_PAD,IRIVER_H10_PAD%
      ,GIGABEAT_PAD,GIGABEAT_S_PAD,MROBE100_PAD,SANSA_E200_PAD,PBELL_VIBE500_PAD%
      ,SANSA_FUZEPLUS_PAD,SAMSUNG_YH92X_PAD,SAMSUNG_YH820_PAD}
    { or by pressing \ActionKbdMorseInput{} in the virtual keyboard}%
  .}

\nopt{player}{% no "Actions" yet in the Player's virtual keyboard

\note{When the cursor is on the input line, \ActionKbdSelect{} deletes the preceding character}

\begin{btnmap}
    \opt{IRIVER_H100_PAD,IRIVER_H300_PAD,RECORDER_PAD,GIGABEAT_PAD,GIGABEAT_S_PAD%
        ,MROBE100_PAD,SANSA_E200_PAD,SANSA_FUZE_PAD,SANSA_C200_PAD,SANSA_FUZEPLUS_PAD%
        ,SAMSUNG_YH820_PAD}{
        \ActionKbdCursorLeft{} / \ActionKbdCursorRight
            &
        \opt{HAVEREMOTEKEYMAP}{\ActionRCKbdCursorLeft{} / \ActionRCKbdCursorRight
            &}
        Move the line cursor within the text line.
            \\
        %
        \ActionKbdBackSpace
            &
        \opt{HAVEREMOTEKEYMAP}{
            &}
        Delete the character before the line cursor.
            \\
    }%
    \ActionKbdLeft{} / \ActionKbdRight
        &
    \opt{HAVEREMOTEKEYMAP}{\ActionRCKbdLeft{} / \ActionRCKbdRight
        &}
    Move the cursor on the virtual keyboard.
    If you move out of the picker area, you get the previous/next page of
    characters (if there is more than one).
        \\
    %
    \ActionKbdUp{} / \ActionKbdDown
        &
    \opt{HAVEREMOTEKEYMAP}{\ActionRCKbdUp{} / \ActionRCKbdDown
        &}
    Move the cursor on the virtual keyboard.
    If you move out of the picker area you get to the line edit mode.
        \\
    %
    \nopt{IPOD_3G_PAD,IPOD_4G_PAD,IRIVER_H10_PAD,ONDIO_PAD,PBELL_VIBE500_PAD%
         ,SANSA_FUZEPLUS_PAD,SAMSUNG_YH92X_PAD,SAMSUNG_YH820_PAD}{
        \ActionKbdPageFlip
            &
        \opt{HAVEREMOTEKEYMAP}{\ActionRCKbdPageFlip
            &}
        Flip to the next page of characters (if there is more than one).
            \\
    }
    %
    \ActionKbdSelect
        &
    \opt{HAVEREMOTEKEYMAP}{\ActionRCKbdSelect
        &}
    Insert the selected keyboard letter at the current line cursor position.
        \\
    %
    \ActionKbdDone
        &
    \opt{HAVEREMOTEKEYMAP}{\ActionRCKbdDone
        &}
    Exit the virtual keyboard and save any changes.
        \\
    %
    \ActionKbdAbort
        &
    \opt{HAVEREMOTEKEYMAP}{\ActionRCKbdAbort
        &}
    Exit the virtual keyboard without saving any changes.
        \\
% to be done - create a separate section for morse imput and update the info
      \opt{morse_input}{
        \opt{IRIVER_H100_PAD,IRIVER_H300_PAD,GIGABEAT_PAD,GIGABEAT_S_PAD,MROBE100_PADD%
            ,SANSA_E200_PA,IPOD_4G_PAD,IPOD_3G_PAD,IRIVER_H10_PAD,PBELL_VIBE500_PAD%
            ,SAMSUNG_YH92X_PAD,SAMSUNG_YH820_PAD}{
          \ActionKbdMorseInput
          \opt{HAVEREMOTEKEYMAP}{& \ActionRCKbdMorseInput}
          & Toggle keyboard input mode and Morse code input mode. \\}
        %
        \ActionKbdMorseSelect
        \opt{HAVEREMOTEKEYMAP}{& \ActionRCKbdMorseSelect}
        & Tap to select a character in Morse code input mode. \\
      } 
\end{btnmap}
}% end of non-Player section

\opt{player}{
  The current text line to be entered or edited is always listed on the first
  line of the display. The second line of the display can contain the character
  selection bar, as in the screenshot above.
    \begin{btnmap}
      \ButtonOn & Toggle picker- and line edit mode. \\
      \ButtonLeft{} / \ButtonRight
        & Move back and forth in the selected line (picker of input line). \\
      \ButtonPlay
        & Pick character in character bar, or act as backspace in the text line. \\
      Long \ButtonPlay & Accept \\
      \ButtonStop & Cancel \\
      \ButtonMenu & Flip picker lines. \\
    \end{btnmap}
}

\input{rockbox_interface/tagcache.tex}
% $Id$ %
\section{\label{ref:WPS}While Playing Screen}
The While Playing Screen (WPS) displays various pieces of information about the
currently playing audio file.
%
\opt{lcd_bitmap}{%
  The appearance of the WPS can be configured using WPS configuration files.
  The items shown depend on your configuration -- all items can be turned on
  or off independently. Refer to \reference{ref:wps_tags} for details on how
  to change the display of the WPS.
  \begin{itemize}
  \item Status bar: The Status bar shows Battery level, charger status,
    volume, play mode, repeat mode, shuffle mode\opt{rtc}{ and clock}.
    In contrast to all other items, the status bar is always at the top of
    the screen.
  \item (Scrolling) path and filename of the current song.
  \item The ID3 track name.
  \item The ID3 album name.
  \item The ID3 artist name.
  \item Bit rate. VBR files display average bitrate and ``(avg)''
  \item Elapsed and total time.
  \item A slidebar progress meter representing where in the song you are.
  \item Peak meter.
  \end{itemize}
}
%

See \reference{ref:ConfiguringtheWPS} for details of customising
your WPS (While Playing Screen).


\subsection{\label{ref:WPS_Key_Controls}WPS Key Controls}

  \begin{btnmap}
      \ActionWpsVolUp{} / \ActionWpsVolDown
      \opt{HAVEREMOTEKEYMAP}{& \ActionRCWpsVolUp{} / \ActionRCWpsVolDown}
      & Volume up/down.\\
      %
      \ActionWpsSkipPrev
       \opt{HAVEREMOTEKEYMAP}{& \ActionRCWpsSkipPrev}
      & Go to beginning of track, or if pressed while in the
        first seconds of a track, go to the previous track.\\
      %
      \ActionWpsSeekBack
      \opt{HAVEREMOTEKEYMAP}{& \ActionRCWpsSeekBack}
      & Rewind in track.\\
      %
      \ActionWpsSkipNext
      \opt{HAVEREMOTEKEYMAP}{& \ActionRCWpsSkipNext}
      & Go to the next track.\\
      %
      \ActionWpsSeekFwd
      \opt{HAVEREMOTEKEYMAP}{& \ActionRCWpsSeekFwd}
      & Fast forward in track.\\
      %
      \ActionWpsPlay
      \opt{HAVEREMOTEKEYMAP}{& \ActionRCWpsPlay}
      & Toggle play/pause.\\
      %
      \ActionWpsStop
      \opt{HAVEREMOTEKEYMAP}{& \ActionRCWpsStop}
      & Stop playback.\\
      %
      \ActionWpsBrowse
      \opt{HAVEREMOTEKEYMAP}{& \ActionRCWpsBrowse}
      & Return to the \setting{File Browser} / \setting{Database}.\\
      %
      \ActionWpsContext
      \opt{HAVEREMOTEKEYMAP}{& \ActionRCWpsContext}
      & Enter \setting{WPS Context Menu}.\\
      %
      \ActionWpsMenu
      \opt{HAVEREMOTEKEYMAP}{& \ActionRCWpsMenu}
      & Enter \setting{Main Menu}%
      .\\%
      %
      \opt{quickscreen}{%
        \ActionWpsQuickScreen
        \opt{HAVEREMOTEKEYMAP}{& \ActionRCWpsQuickScreen}
          & Switch to the \setting{Quick Screen}
          (see \reference{ref:QuickScreen}). \\}%
      %
      % software hold targets
      \nopt{hold_button}{%
          \opt{SANSA_CLIP_PAD}{\ButtonHome+\ButtonSelect}
          \opt{SANSA_FUZEPLUS_PAD}{\ButtonPower}
          & Key lock (software hold switch) on/off.\\
      }%
      % We explicitly list all the appropriate targets here and do no condition
      % on the 'pitchscreen' feature since some players have the feature but do
      % not have the button to go from the WPS to the pitch screen.
      \opt{IRIVER_H100_PAD,IRIVER_H300_PAD,IRIVER_H10_PAD,MROBE100_PAD%
          ,GIGABEAT_PAD,GIGABEAT_S_PAD,SANSA_E200_PAD,SANSA_C200_PAD,SANSA_FUZEPLUS_PAD}{%
        \ActionWpsPitchScreen
        \opt{HAVEREMOTEKEYMAP}{& \ActionRCWpsPitchScreen}
          & Show \setting{Pitch Screen} (see \reference{sec:pitchscreen}).\\%
      }%
      \opt{GIGABEAT_PAD,GIGABEAT_S_PAD,SANSA_CLIP_PAD,MROBE100_PAD,PBELL_VIBE500_PAD%
          ,SAMSUNG_YH92X_PAD,SAMSUNG_YH820_PAD}{%
        \ActionWpsPlaylist
        \opt{HAVEREMOTEKEYMAP}{&}
          & Show current \setting{Playlist}.\\%
      }%
      \opt{IRIVER_H100_PAD,IRIVER_H300_PAD,IRIVER_H10_PAD%
          ,SANSA_E200_PAD,SANSA_C200_PAD,SANSA_FUZEPLUS_PAD}{%
        \ActionWpsIdThreeScreen
          \opt{HAVEREMOTEKEYMAP}{& \ActionRCWpsIdThreeScreen}
          & Enter \setting{ID3 Viewer}.\\%
      }%
      \opt{hotkey}{%
        \ActionWpsHotkey \opt{HAVEREMOTEKEYMAP}{& }
        & Activate the \setting{Hotkey} function (see \reference{ref:Hotkeys}).\\
      }
      \opt{ab_repeat_buttons}{%
         \ActionWpsAbSetBNextDir{} or }%
         % not all targets have the above action defined but the one below works on all
      Short \ActionWpsSkipNext{} + Long \ActionWpsSkipNext
      \opt{HAVEREMOTEKEYMAP}{
        &
          \opt{IRIVER_RC_H100_PAD}{\ActionRCWpsAbSetBNextDir{} or}
        Short \ActionRCWpsSkipNext{} + Long \ActionRCWpsSkipNext}
      & Skip to the next directory.\\
      %
      \opt{ab_repeat_buttons}{%
         \ActionWpsAbSetAPrevDir{} or }%
      Short \ActionWpsSkipPrev{} + Long \ActionWpsSkipPrev
      \opt{HAVEREMOTEKEYMAP}{
        &
          \opt{IRIVER_RC_H100_PAD}{\ActionRCWpsAbSetAPrevDir{} or}
        Short \ActionRCWpsSkipPrev{} + Long \ActionRCWpsSkipPrev}
      & Skip to the previous directory.\\
      %
      \opt{SANSA_E200_PAD,SANSA_C200_PAD,IRIVER_H100_PAD,IRIVER_H300_PAD}{
        \ActionStdRec
          \opt{HAVEREMOTEKEYMAP}{&}
          & Switch to the \setting{Recording Screen}.\\
      }%
  \end{btnmap}


\opt{lcd_bitmap}{
  \subsection{\label{ref:peak_meter}Peak Meter}
  The peak meter can be displayed on the While Playing Screen and consists of
  several indicators.
  \opt{recording}{
    For a picture of the peak meter, please see the While
    Recording Screen in \reference{ref:while_recording_screen}.
  }
  \opt{ipodvideo}{
    \note{Especially the \playerman{} \playertype{}'s performance and battery runtime
     suffers when this feature is enabled. For this \dap{} it is highly recommended
     to not use peak meter.}
  }

  \begin{description}
  \item [The bar:]
    This is the wide horizontal bar. It represents the current volume value.
  \item [The peak indicator:]
    This is a little vertical line at the right end of the bar. It indicates
    the peak volume value that occurred recently.
  \item [The clip indicator:]
    This is a little black block that is displayed at the very right of the
    scale when an overflow occurs. It usually does not show up during normal
    playback unless you play an audio file that is distorted heavily.
    \opt{recording}{
      If you encounter clipping while recording, your recording will sound distorted.
      You should lower the gain.
    }
    \note{Note that the clip detection is not very precise.
     Clipping might occur without being indicated.}
  \item [The scale:]
    Between the indicators of the right and left channel there are little dots.
    These dots represent important volume values. In linear mode each dot is a
    10\% mark. In dBFS mode the dots represent the following values (from right
    to left): 0~dB, {}-3~dB, {}-6~dB, {}-9~dB, {}-12~dB, {}-18~dB, {}-24~dB, {}-30~dB,
    {}-40~dB, {}-50~dB, {}-60~dB.
  \end{description}
}
\subsection{\label{sec:contextmenu}The WPS Context Menu}
Like the context menu for the \setting{File Browser}, the \setting{WPS Context Menu}
allows you quick access to some often used functions.

\subsubsection{Playlist}
The \setting{Playlist} submenu allows you to view, save, search, reshuffle,
and display the play time of the current playlist. These and other operations
are detailed in \reference{ref:working_with_playlists}. To change settings for
the \setting{Playlist Viewer} press \ActionStdContext{} while viewing the
current playlist to bring up the \setting{Playlist Viewer Menu}. In this
menu, you can find the \setting{Playlist Viewer Settings}.

\paragraph{Playlist Viewer Settings}
  \begin{description}
    \item[Show Icons.] This toggles display of the icon for the currently
    selected playlist entry and the icon for moving a playlist entry
    \item[Show Indices.] This toggles display of the line numbering for
       the playlist
    \item[Track Display.] This toggles between filename only and full path
       for playlist entries
  \end{description}


\subsubsection{Playlist catalogue}
  \begin{description}
    \item [View catalogue.] This lists all playlists that are part of the
    Playlist catalogue. You can load a new playlist directly from this list.
    \item [Add to playlist.] Adds the currently playing file to a playlist.
    Select the playlist you want the file to be added to and it will get
    appended to that playlist.
    \item [Add to new playlist.] Similar to the previous entry this will
    add the currently playing track to a playlist. You need to enter a name
    for the new playlist first.
  \end{description}

\subsubsection{Sound Settings}
This is a shortcut to the \setting{Sound Settings Menu}, where you can configure volume,
bass, treble, and other settings affecting the sound of your music.
See \reference{ref:configure_rockbox_sound} for more information.

\subsubsection{Playback Settings}
This is a shortcut to the \setting{Playback Settings Menu}, where you can configure shuffle,
repeat, party mode, skip length and other settings affecting the playback of your music.

\subsubsection{Rating}
The menu entry is only shown if \setting{Gather Runtime Information} is
enabled. It allows the assignment of a personal rating value (0 -- 10)
to a track which can be displayed in the WPS and used in the Database
browser. The value wraps at 10.

\subsubsection{Bookmarks}
This allows you to create a bookmark in the currently-playing track.

\subsubsection{\label{ref:trackinfoviewer}Show Track Info}
\screenshot{rockbox_interface/images/ss-id3-viewer}{The track info viewer}{}
This screen is accessible from the WPS screen, and provides a detailed view of
all the identity information about the current track. This info is known as
meta data and is stored in audio file formats to keep information on artist,
album etc. To access this screen, %
\opt{IRIVER_H100_PAD,IRIVER_H300_PAD,IRIVER_H10_PAD,%
      SANSA_C200_PAD,SANSA_E200_PAD,SANSA_FUZE_PAD,SANSA_FUZEPLUS_PAD}{
  press \ActionWpsIdThreeScreen. }%
\opt{IPOD_4G_PAD,IPOD_3G_PAD,IAUDIO_X5_PAD,IAUDIO_M3_PAD,%
      GIGABEAT_PAD,GIGABEAT_S_PAD,MROBE100_PAD,SANSA_CLIP_PAD,PBELL_VIBE500_PAD,%
      MPIO_HD200_PAD,MPIO_HD300_PAD,SAMSUNG_YH92X_PAD,SAMSUNG_YH820_PAD}%
      {press \ActionWpsContext{} to access the
      \setting{WPS Context Menu} and select \setting{Show Track Info}. }

\subsubsection{Open With...}
This \setting{Open With} function is the same as the \setting{Open With}
function in the file browser's \setting{Context Menu}.

\subsubsection{Delete}
Delete the currently playing file. The file will be deleted but the playback
of the file will not stop immediately. Instead, the part of the file that
has already been buffered (i.e. read into the \daps\ memory) will be played.
This may even be the whole track.

\opt{pitchscreen}{
  \subsubsection{\label{sec:pitchscreen}Pitch}

  The \setting{Pitch Screen} allows you to change the rate of playback
  (i.e. the playback speed and at the same time the pitch) of your
  \dap.  The rate value can be adjusted
  between 50\% and 200\%. 50\% means half the normal playback speed and a
  pitch that is an octave lower than the normal pitch. 200\% means double
  playback speed and a pitch that is an octave higher than the normal pitch.

  The rate can be changed in two modes: procentual and semitone.
  Initially, procentual mode is active.

  \opt{swcodec}{
    If you've enabled the \setting{Timestretch} option in
    \setting{Sound Settings} and have since rebooted, you can also use
    timestretch mode. This allows you to change the playback speed
    without affecting the pitch, and vice versa.

    In timestretch mode there are separate displays for pitch and
    speed, and each can be altered independently.  Due to the
    limitations of the algorithm, speed is limited to be between 35\%
    and 250\% of the current pitch value.  Pitch must maintain the
    same ratio as well as remain between 50\% and 200\%.
  }

  The value of the \opt{swcodec}{rate, pitch and speed is persistent,
  i.e. it is saved and restored}
  \nopt{swcodec}{rate is not persistent, i.e.}
  after the \dap\ is turned on
  \nopt{swcodec}{it will always be set to 100\%}.
  \opt{swcodec}{ Moreover, the rate, pitch and speed
  information will be stored in any bookmarks you may create
  (see \reference{ref:Bookmarkconfigactual}) and will be restored upon
  playing back those bookmarks.}

  \nopt{swcodec}{
      \begin{btnmap}
        \ActionPsToggleMode
        & Toggle pitch changing mode. \\
        %
        \ActionPsIncSmall{} / \ActionPsDecSmall
        & Increase~/ Decrease pitch by 0.1\% (in procentual mode) or by 0.1
          semitone (in semitone mode).\\
        %
        \ActionPsIncBig{} / \ActionPsDecBig
        & Increase~/ Decrease pitch by 1\% (in procentual mode) or a semitone
          (in semitone mode).\\
        %
        \ActionPsNudgeLeft{} / \ActionPsNudgeRight
        & Temporarily change pitch by 2\% (beatmatch). \\
        %
        \ActionPsReset
        & Reset rate to 100\%. \\
        %
        \ActionPsExit
        & Leave the \setting{Pitch Screen}. \\
        %
      \end{btnmap}

    \warn{Changing the pitch can cause audible `Artifacts' or `Dropouts'.}
  }

  \opt{swcodec}{
      \begin{btnmap}
        \ActionPsToggleMode
        \opt{HAVEREMOTEKEYMAP}{& \ActionRCPsToggleMode}
        & Toggle pitch changing mode (cycle through all available modes).\\
        %
        \ActionPsIncSmall{} / \ActionPsDecSmall
        \opt{HAVEREMOTEKEYMAP}{& \ActionRCPsIncSmall{} / \ActionRCPsDecSmall}
        & Increase~/ Decrease pitch by 0.1\% (in procentual mode) or 0.1
          semitone (in semitone mode).\\
        %
        \nopt{PBELL_VIBE500_PAD}{ % there is no long scroll up or down because of slide
        \ActionPsIncBig{} / \ActionPsDecBig
        \opt{HAVEREMOTEKEYMAP}{& \ActionRCPsIncBig{} / \ActionRCPsDecBig}
        & Increase~/ Decrease pitch by 1\% (in procentual mode) or a semitone
          (in semitone mode).\\
        }
        %
        \ActionPsNudgeLeft{} / \ActionPsNudgeRight
        \opt{HAVEREMOTEKEYMAP}{& \ActionRCPsNudgeLeft{} / \ActionRCPsNudgeRight}
        & Temporarily change pitch by 2\% (beatmatch), or modify speed (in timestretch mode).\\
        %
        \ActionPsReset
        \opt{HAVEREMOTEKEYMAP}{& \ActionRCPsReset}
        & Reset pitch and speed to 100\%. \\
        %
        \ActionPsExit
        \opt{HAVEREMOTEKEYMAP}{& \ActionRCPsExit}
        & Leave the \setting{Pitch Screen}. \\
        %
      \end{btnmap}
  }

}


%Include playlist section
% $Id$ %
\chapter{The Main Menu}
\section{\label{ref:main_menu}Introducing the Main Menu}
\screenshot{main_menu/images/ss-main-menu}{The main menu}{}
The \setting{Main Menu} is the screen from which all of the Rockbox functions
can be accessed. This is the first screen you will see when starting Rockbox.
To return to the \setting{Main Menu}, 
  \nopt{ONDIO_PAD}{press the \ActionStdMenu{} button.}%
  \opt{ONDIO_PAD}{hold the \ButtonMenu{} button.}%

All settings are stored on the unit. However, Rockbox does not access 
the \disk{} solely for the purpose of saving settings. Instead, Rockbox will
save settings when it accesses the \disk{} the next time, for example when 
refilling the music buffer or navigating through the \setting{File Browser}.
Changes to settings may therefore not be saved unless the \dap{} is shut down
safely (see \reference{ref:Safeshutdown}).

\section{Navigating the Main Menu}
  \begin{btnmap}
    \ActionStdNext
        &
    \opt{HAVEREMOTEKEYMAP}{\ActionRCStdNext
        &}
    Select the next option in the menu.\newline
    Inside a setting, increase the value or choose next option.
        \\
    %
    \ActionStdPrev
        &
    \opt{HAVEREMOTEKEYMAP}{\ActionRCStdPrev
        &}
    Select the previous option in the menu.\newline
    Inside a setting,decrease the value or choose previous option.
        \\
    %
    \ActionStdOk
        &
    \opt{HAVEREMOTEKEYMAP}{\ActionRCStdOk
        &}
    Select option.
        \\
    %
    \ActionStdCancel
        &
    \opt{HAVEREMOTEKEYMAP}{\ActionRCStdCancel
        &}
    Exit menu or setting, or move to parent menu.
        \\
  \end{btnmap}

\section {Recent Bookmarks}
\screenshot{main_menu/images/ss-list-bookmarks}%
{The list bookmarks screen}{}
If the \setting{Save a list of recently created bookmarks} option is enabled 
then you can view a list of several recent bookmarks here and select one to 
jump straight to that track.\\*

 \note{Bookmarking only works when tracks are launched from the file browser,
        and does not currently work for tracks launched via the
        database. In addition, they do not currently work with dynamic
        playlists.\\*} 

  \begin{btnmap}
    \ActionStdNext
    \opt{HAVEREMOTEKEYMAP}{& \ActionRCStdNext}
    & Select the next bookmark.\\
    %
    \ActionStdPrev
    \opt{HAVEREMOTEKEYMAP}{& \ActionRCStdPrev}
    & Select the previous bookmark.\\
    %
    \ActionStdOk
    \opt{HAVEREMOTEKEYMAP}{& \ActionRCStdOk}
    & Resume from the selected bookmark.\\
    %
    \ActionStdCancel
    \opt{HAVEREMOTEKEYMAP}{& \ActionRCStdCancel}
    & Exit Recent Bookmark menu.\\
    %
    \nopt{GIGABEAT_S_PAD}{\ActionBmDelete
    \opt{HAVEREMOTEKEYMAP}{& \ActionRCBmDelete}
    & Delete the currently selected bookmark.\\}
    %
    \ActionStdContext
    \opt{HAVEREMOTEKEYMAP}{& \ActionRCStdContext}  
    & Enter the context menu for the selected bookmark.\\
  \end{btnmap}

There are two options in the context menu:\\*
  
  \setting{Resume} will commence playback of the currently selected bookmark entry.
  
  \setting{Delete} will remove the currently selected bookmark entry from the list.\\*
  
This entry is not shown in the \setting{Main Menu} when the option is off
(the default setting).  See \reference{ref:Bookmarkconfigactual} 
for more details on configuring bookmarking in Rockbox.

\section{Files}
Browse the files on your \dap{} (see \reference{ref:file_browser}).

\section{Database}
Browse by the meta-data in your audio files (see \reference{ref:database}).

\section{Now Playing/Resume Playback}
Go to the \setting{While Playing Screen} and resume if music playback is
stopped or paused and there is something to resume (see \reference{ref:WPS}).

\section{Settings}

The \setting{Settings} menu allows you to set or adjust many parameters that
affect the way your \dap{} works. There are many submenus for different
parameter areas. Every time you are setting a value of a parameter, and that
value is selected from a list of some predefined available values, you can press
\ActionStdContext, and the selection cursor will jump to the default value for
the parameter. You can then confirm or cancel the value. This is useful if you
have changed the value of the parameter from the default to some other value and
would like to restore the default value.

\subsection{Sound Settings}
The \setting{Sound Settings} menu offers a selection of sound properties you may 
change to customise your listening experience. The details of this menu are covered
in \reference{ref:configure_rockbox_sound}.

\subsection{Playback Settings}
The \setting{Playback Settings} menu allows you to configure settings related
to audio playback. The details of this menu are covered
in \reference{ref:configure_rockbox_playback}.

\subsection{General Settings}
The \setting{General Settings} menu allows you to customise the way Rockbox looks 
and the way it plays music. The details of this menu are covered in
\reference{ref:configure_rockbox_general}.

\subsection{Theme Settings}
The \setting{Theme Settings} menu contains options that control the visual
appearance of Rockbox. The details of this menu are covered in
\reference{ref:configure_rockbox_themes}.

\opt{recording}{
\subsection{Recording Settings}
The \setting{Recording Settings} menu allows you to configure settings related
to recording. The details of this menu are covered in detail in
\reference{ref:Recordingsettings}.
}

\subsection{Manage Settings}
The \setting{Manage Settings} option allows the saving and re-loading of user 
configuration settings, browsing the hard drive for alternate firmwares, and finally
resetting your \dap{} back to initial configuration.
%
The details of this menu are covered in
\reference{ref:manage_settings}.

\opt{recording}{\input{main_menu/recording_screen.tex}}

\opt{radio}{\input{main_menu/fmradio.tex}}

\section{\label{ref:playlistoptions}Playlists}
  This menu allows you to work with playlists. Playlists can be created in 
  three ways. Playing a file in a directory causes all the files in it
  to be placed in a playlist. Playlists can be created manually by
  either using the  \setting{Context Menu} (see \reference{ref:Contextmenu}) or using
  the \setting{Playlist} menu. Both automatically and manually created
  playlists can be edited using this menu.

\begin{description}
\item[Create Playlist:]
  Rockbox will create a playlist with all tracks in the current directory 
and all sub-directories. The playlist will be created one directory level ``up'' 
from where you currently are.
  
\item[View Current Playlist:]
  Displays the contents of the playlist currently stored in memory.
  
\item[Save Current Playlist:]
  Saves the current dynamic playlist, excluding queued tracks, to the 
specified file. If no path is provided then playlist is saved to the current 
directory.

\item[View Catalogue:]
  Provides a simple interface to maintain
  several playlists (see \reference{ref:working_with_playlists}).
\end{description}

\section{Plugins}
  With this option you can load and run various plugins that have been
written for Rockbox. There are a wide variety of these supplied with
Rockbox, including several games, some impressive demos and a number of
utilities. A detailed description of the different plugins is to be found in 
\reference{ref:plugins}.

\section{\label{ref:Info}System}
\opt{player}{Use the MINUS and PLUS keys to step through several 
pages of information.}

\begin{description}
\item[Rockbox Info:]
  Displays some basic system information. This is, from top to bottom,
  the amount of memory Rockbox has available for storing music (the buffer).
  The battery status.
\opt{multivolume}{%
  Memory size and amount of free space on the two data volumes, this info is
  given separately for internal memory (\emph{Int}) and for a plugged in
  memory card
  \opt{ondio}{(\emph{MMC})}
  \opt{sansa,e200v2,fuze,fuzev2,clipplus,clipzip}{(\emph{MSD})}.
}%
\nopt{multivolume}{Hard disk size and the amount of free space on the disk.}

\item[Credits:]
  Display the list of contributors.

\item[Running Time:]
  Shows the runtime of your \dap{} in hours, minutes and seconds.
  \begin{description}
    \item[Running Time:]
        This item shows the cumulative overall runtime of your \dap{} since you 
        either disconnected it from charging (in Rockbox) or manually 
        reset this item. A manual reset is done through pressing any button, 
        followed by pressing \ActionStdOk{}.
    \item[Top Time:]
        This item shows the cumulative overall runtime of your \dap{} since you 
        last manually reset this item. A manual reset is done through pressing 
        any button, followed by pressing \ActionStdOk{}.
  \end{description}
\end{description}

\opt{player}{
  \section{Shutdown}
  This menu option saves the Rockbox configuration and turns off the hard
  drive before shutting down the machine. For maximum safety this procedure
  is recommended when turning off the \dap. (There is a very small risk
  of hard disk corruption otherwise.) See \reference{ref:Safeshutdown}
  for more details.
}

\opt{quickscreen}
{
\section{\label{ref:QuickScreen}Quick Screen}
  Although the \setting{Quick Screen} is accessible from nearly everywhere,
  not just the \setting{Main Menu}, it is worth mentioning here.  It allows
  rapid access to your four favourite settings.  The default settings are
  \setting{Shuffle} (\reference{ref:PlaybackSettings}),
  \setting{Repeat} (\reference{ref:PlaybackSettings}) and the
  \setting{Show Files} (\reference{ref:ShowFiles}) options, but almost all
  configurable options in Rockbox can be placed on this screen.  To change the
  options, navigate through the menus to the setting you want to add and press
  \ActionStdContext.  In the menu which appears you will be given options
  to place the setting on the \setting{Quick Screen}.
  
  Press \ActionStdQuickScreen{} to access it and \ActionQuickScreenExit{} to exit.
  The direction buttons will modify the individual setting values as indicated 
  by the arrow icons. Please note that the settings at opposite sides of the
   screen cycle through the available options in opposite directions.
   Therefore if you select the same setting at e.g. the top and bottom of the
   quickscreen, then pressing up and down will cycle through this setting in
   opposite directions.
}

\section{\label{ref:MainMenuShortcuts}Shortcuts}

This menu item is a container for user defined shortcuts to files, folders or
settings. With a shortcut,
\begin{itemize}
  \item A file can be ``run'' (i.e. a music file played, plugin started or
        a \fname{.cfg} loaded)
  \item The file browser can be opened with the cursor positioned at
        a specified file or folder
  \item A file's or folder's ``Current Playlist'' context menu item can
        be displayed
  \item A setting can be configured (any which can be added to the
        \setting{Quick Screen})
\opt{rtc}{
  \item The current time can be spoken
}
  \item The sleep timer can be configured
  \item The \dap{} can be turned off
\end{itemize}

\note{Shortcuts into the database are not possible}

Shortcuts are loaded from the file \fname{/.rockbox/shortcuts.txt} which lists
each item to be displayed. Each shortcut looks like the following:

\begin{example}
    [shortcut]
    type: <shortcut type>
    data: <what the shortcut actually links to>
    name: <what the shortcut should be displayed as>
    icon: <number of the theme icon to use (see \wikilink{CustomIcons})>
    talkclip: <filename of a talk clip to speak when voice menus are enabled>
\end{example}

Only ``type'' and ``data'' are required (except if type is ``separator'' in which case
``data'' is also not required).

Available types are:
\begin{description}
\item[file] \config{data} is the name of the file to ``run''
\item[browse] \config{data} is the file or the folder to open the file browser at
\item[playlist menu] \config{data} is the file or the folder to open the
  ``Current Playlist'' context menu item on
\item[setting] \config{data} is the config name of the setting you want to change
  (see \reference{ref:config_file_options} for the list of the possible settings)
\item[separator] \config{data} is ignored; \config{name} can be used to display text,
  or left blank to make the list more accessible with visual gaps
\item[time] \config{data} needs to be \opt{rtc}{either ``talk'' to talk the time, or }``sleep X''
  where X is the number of minutes to run the sleep timer for (0 to disable). \config{name}
  is required for this shortcut type.
\item[shutdown] \config{data} is ignored; \config{name} can be used to display text
\end{description}

If the name/icon items are not specified, a sensible default will be used.

\note{For the ``browse'' type, if you want the file browser to start \emph{inside}
a folder, make sure the data has the trailing slash (i.e \fname{/Music/} instead of
\fname {/Music}). Without the trailing slash, it will cause the file browser to open
with \fname{/Music} selected instead.}

The file \fname{shortcuts.txt} can be edited with any text editor. Most items can
also be added to it through their context menu item ``Add to shortcuts''.
A reboot is needed for manual changes to \fname{shortcuts.txt} to be applied.

Shortcuts can be manually removed by selecting the one you wish to remove and pressing
\ActionStdContext{}.

\input{rockbox_interface/hotkeys.tex}
