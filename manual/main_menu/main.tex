% $Id$ %
\chapter{The Main Menu}
\section{\label{ref:main_menu}Introducing the Main Menu}
\screenshot{main_menu/images/ss-main-menu}{The main menu}{}
The \setting{Main Menu} is the screen from which all of the Rockbox functions
can be accessed. This is the first screen you will see when starting Rockbox.
To return to the \setting{Main Menu}, 
  \nopt{ONDIO_PAD}{press the \ActionStdMenu{} button.}%
  \opt{ONDIO_PAD}{hold the \ButtonMenu{} button.}%

All settings are stored on the unit. However, Rockbox does not access 
the \disk{} solely for the purpose of saving settings. Instead, Rockbox will
save settings when it accesses the \disk{} the next time, for example when 
refilling the music buffer or navigating through the \setting{File Browser}.
Changes to settings may therefore not be saved unless the \dap{} is shut down
safely (see \reference{ref:Safeshutdown}).

\section{Navigating the Main Menu}
  \begin{btnmap}
    \ActionStdNext
        &
    \opt{HAVEREMOTEKEYMAP}{\ActionRCStdNext
        &}
    Select the next option in the menu.\newline
    Inside a setting, increase the value or choose next option.
        \\
    %
    \ActionStdPrev
        &
    \opt{HAVEREMOTEKEYMAP}{\ActionRCStdPrev
        &}
    Select the previous option in the menu.\newline
    Inside a setting,decrease the value or choose previous option.
        \\
    %
    \ActionStdOk
        &
    \opt{HAVEREMOTEKEYMAP}{\ActionRCStdOk
        &}
    Select option.
        \\
    %
    \ActionStdCancel
        &
    \opt{HAVEREMOTEKEYMAP}{\ActionRCStdCancel
        &}
    Exit menu or setting, or move to parent menu.
        \\
  \end{btnmap}

\section {Recent Bookmarks}
\screenshot{main_menu/images/ss-list-bookmarks}%
{The list bookmarks screen}{}
If the \setting{Save a list of recently created bookmarks} option is enabled 
then you can view a list of several recent bookmarks here and select one to 
jump straight to that track.\\*

 \note{Bookmarking only works when tracks are launched from the file browser,
        and does not currently work for tracks launched via the
        database. In addition, they do not currently work with dynamic
        playlists.\\*} 

  \begin{btnmap}
    \ActionStdNext
    \opt{HAVEREMOTEKEYMAP}{& \ActionRCStdNext}
    & Select the next bookmark.\\
    %
    \ActionStdPrev
    \opt{HAVEREMOTEKEYMAP}{& \ActionRCStdPrev}
    & Select the previous bookmark.\\
    %
    \ActionStdOk
    \opt{HAVEREMOTEKEYMAP}{& \ActionRCStdOk}
    & Resume from the selected bookmark.\\
    %
    \ActionStdCancel
    \opt{HAVEREMOTEKEYMAP}{& \ActionRCStdCancel}
    & Exit Recent Bookmark menu.\\
    %
    \nopt{GIGABEAT_S_PAD}{\ActionBmDelete
    \opt{HAVEREMOTEKEYMAP}{& \ActionRCBmDelete}
    & Delete the currently selected bookmark.\\}
    %
    \ActionStdContext
    \opt{HAVEREMOTEKEYMAP}{& \ActionRCStdContext}  
    & Enter the context menu for the selected bookmark.\\
  \end{btnmap}

There are two options in the context menu:\\*
  
  \setting{Resume} will commence playback of the currently selected bookmark entry.
  
  \setting{Delete} will remove the currently selected bookmark entry from the list.\\*
  
This entry is not shown in the \setting{Main Menu} when the option is off
(the default setting).  See \reference{ref:Bookmarkconfigactual} 
for more details on configuring bookmarking in Rockbox.

\section{Files}
Browse the files on your \dap{} (see \reference{ref:file_browser}).

\section{Database}
Browse by the meta-data in your audio files (see \reference{ref:database}).

\section{Now Playing/Resume Playback}
Go to the \setting{While Playing Screen} and resume if music playback is
stopped or paused and there is something to resume (see \reference{ref:WPS}).

\section{Settings}

The \setting{Settings} menu allows you to set or adjust many parameters that
affect the way your \dap{} works. There are many submenus for different
parameter areas. Every time you are setting a value of a parameter, and that
value is selected from a list of some predefined available values, you can press
\ActionStdContext, and the selection cursor will jump to the default value for
the parameter. You can then confirm or cancel the value. This is useful if you
have changed the value of the parameter from the default to some other value and
would like to restore the default value.

\subsection{Sound Settings}
The \setting{Sound Settings} menu offers a selection of sound properties you may 
change to customise your listening experience. The details of this menu are covered
in \reference{ref:configure_rockbox_sound}.

\subsection{Playback Settings}
The \setting{Playback Settings} menu allows you to configure settings related
to audio playback. The details of this menu are covered
in \reference{ref:configure_rockbox_playback}.

\subsection{General Settings}
The \setting{General Settings} menu allows you to customise the way Rockbox looks 
and the way it plays music. The details of this menu are covered in
\reference{ref:configure_rockbox_general}.

\subsection{Theme Settings}
The \setting{Theme Settings} menu contains options that control the visual
appearance of Rockbox. The details of this menu are covered in
\reference{ref:configure_rockbox_themes}.

\opt{recording}{
\subsection{Recording Settings}
The \setting{Recording Settings} menu allows you to configure settings related
to recording. The details of this menu are covered in detail in
\reference{ref:Recordingsettings}.
}

\subsection{Manage Settings}
The \setting{Manage Settings} option allows the saving and re-loading of user 
configuration settings, browsing the hard drive for alternate firmwares, and finally
resetting your \dap{} back to initial configuration.
%
The details of this menu are covered in
\reference{ref:manage_settings}.

\opt{recording}{\input{main_menu/recording_screen.tex}}

\opt{radio}{\input{main_menu/fmradio.tex}}

\section{\label{ref:playlistoptions}Playlists}
  This menu allows you to work with playlists. Playlists can be created in 
  three ways. Playing a file in a directory causes all the files in it
  to be placed in a playlist. Playlists can be created manually by
  either using the  \setting{Context Menu} (see \reference{ref:Contextmenu}) or using
  the \setting{Playlist} menu. Both automatically and manually created
  playlists can be edited using this menu.

\begin{description}
\item[Create Playlist:]
  Rockbox will create a playlist with all tracks in the current directory 
and all sub-directories. The playlist will be created one directory level ``up'' 
from where you currently are.
  
\item[View Current Playlist:]
  Displays the contents of the playlist currently stored in memory.
  
\item[Save Current Playlist:]
  Saves the current dynamic playlist, excluding queued tracks, to the 
specified file. If no path is provided then playlist is saved to the current 
directory.

\item[View Catalogue:]
  Provides a simple interface to maintain
  several playlists (see \reference{ref:working_with_playlists}).
\end{description}

\section{Plugins}
  With this option you can load and run various plugins that have been
written for Rockbox. There are a wide variety of these supplied with
Rockbox, including several games, some impressive demos and a number of
utilities. A detailed description of the different plugins is to be found in 
\reference{ref:plugins}.

\section{\label{ref:Info}System}
\opt{player}{Use the MINUS and PLUS keys to step through several 
pages of information.}

\begin{description}
\item[Rockbox Info:]
  Displays some basic system information. This is, from top to bottom,
  the amount of memory Rockbox has available for storing music (the buffer).
  The battery status.
\opt{multivolume}{%
  Memory size and amount of free space on the two data volumes, this info is
  given separately for internal memory (\emph{Int}) and for a plugged in
  memory card
  \opt{ondio}{(\emph{MMC})}
  \opt{sansa,e200v2,fuze,fuzev2,clipplus,clipzip}{(\emph{MSD})}.
}%
\nopt{multivolume}{Hard disk size and the amount of free space on the disk.}

\item[Credits:]
  Display the list of contributors.

\item[Running Time:]
  Shows the runtime of your \dap{} in hours, minutes and seconds.
  \begin{description}
    \item[Running Time:]
        This item shows the cumulative overall runtime of your \dap{} since you 
        either disconnected it from charging (in Rockbox) or manually 
        reset this item. A manual reset is done through pressing any button, 
        followed by pressing \ActionStdOk{}.
    \item[Top Time:]
        This item shows the cumulative overall runtime of your \dap{} since you 
        last manually reset this item. A manual reset is done through pressing 
        any button, followed by pressing \ActionStdOk{}.
  \end{description}
\end{description}

\opt{player}{
  \section{Shutdown}
  This menu option saves the Rockbox configuration and turns off the hard
  drive before shutting down the machine. For maximum safety this procedure
  is recommended when turning off the \dap. (There is a very small risk
  of hard disk corruption otherwise.) See \reference{ref:Safeshutdown}
  for more details.
}

\opt{quickscreen}
{
\section{\label{ref:QuickScreen}Quick Screen}
  Although the \setting{Quick Screen} is accessible from nearly everywhere,
  not just the \setting{Main Menu}, it is worth mentioning here.  It allows
  rapid access to your four favourite settings.  The default settings are
  \setting{Shuffle} (\reference{ref:PlaybackSettings}),
  \setting{Repeat} (\reference{ref:PlaybackSettings}) and the
  \setting{Show Files} (\reference{ref:ShowFiles}) options, but almost all
  configurable options in Rockbox can be placed on this screen.  To change the
  options, navigate through the menus to the setting you want to add and press
  \ActionStdContext.  In the menu which appears you will be given options
  to place the setting on the \setting{Quick Screen}.
  
  Press \ActionStdQuickScreen{} to access it and \ActionQuickScreenExit{} to exit.
  The direction buttons will modify the individual setting values as indicated 
  by the arrow icons. Please note that the settings at opposite sides of the
   screen cycle through the available options in opposite directions.
   Therefore if you select the same setting at e.g. the top and bottom of the
   quickscreen, then pressing up and down will cycle through this setting in
   opposite directions.
}

\section{\label{ref:MainMenuShortcuts}Shortcuts}

This menu item is a container for user defined shortcuts to files, folders or
settings. With a shortcut,
\begin{itemize}
  \item A file can be ``run'' (i.e. a music file played, plugin started or
        a \fname{.cfg} loaded)
  \item The file browser can be opened with the cursor positioned at
        a specified file or folder
  \item A file's or folder's ``Current Playlist'' context menu item can
        be displayed
  \item A setting can be configured (any which can be added to the
        \setting{Quick Screen})
\opt{rtc}{
  \item The current time can be spoken
}
  \item The sleep timer can be configured
  \item The \dap{} can be turned off
\end{itemize}

\note{Shortcuts into the database are not possible}

Shortcuts are loaded from the file \fname{/.rockbox/shortcuts.txt} which lists
each item to be displayed. Each shortcut looks like the following:

\begin{example}
    [shortcut]
    type: <shortcut type>
    data: <what the shortcut actually links to>
    name: <what the shortcut should be displayed as>
    icon: <number of the theme icon to use (see \wikilink{CustomIcons})>
    talkclip: <filename of a talk clip to speak when voice menus are enabled>
\end{example}

Only ``type'' and ``data'' are required (except if type is ``separator'' in which case
``data'' is also not required).

Available types are:
\begin{description}
\item[file] \config{data} is the name of the file to ``run''
\item[browse] \config{data} is the file or the folder to open the file browser at
\item[playlist menu] \config{data} is the file or the folder to open the
  ``Current Playlist'' context menu item on
\item[setting] \config{data} is the config name of the setting you want to change
  (see \reference{ref:config_file_options} for the list of the possible settings)
\item[separator] \config{data} is ignored; \config{name} can be used to display text,
  or left blank to make the list more accessible with visual gaps
\item[time] \config{data} needs to be \opt{rtc}{either ``talk'' to talk the time, or }``sleep X''
  where X is the number of minutes to run the sleep timer for (0 to disable). \config{name}
  is required for this shortcut type.
\item[shutdown] \config{data} is ignored; \config{name} can be used to display text
\end{description}

If the name/icon items are not specified, a sensible default will be used.

\note{For the ``browse'' type, if you want the file browser to start \emph{inside}
a folder, make sure the data has the trailing slash (i.e \fname{/Music/} instead of
\fname {/Music}). Without the trailing slash, it will cause the file browser to open
with \fname{/Music} selected instead.}

The file \fname{shortcuts.txt} can be edited with any text editor. Most items can
also be added to it through their context menu item ``Add to shortcuts''.
A reboot is needed for manual changes to \fname{shortcuts.txt} to be applied.

Shortcuts can be manually removed by selecting the one you wish to remove and pressing
\ActionStdContext{}.
