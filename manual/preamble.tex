%%%%%%%%%%%%%%%%%%%%%%%%%%%%%%%%%%%%%%%%%%%%%%%%%%%%%%%%%%%%%%%%%%%%%%%%%%%%%
%             __________               __   ___.
%   Open      \______   \ ____   ____ |  | _\_ |__   _______  ___
%   Source     |       _//  _ \_/ ___\|  |/ /| __ \ /  _ \  \/  /
%   Jukebox    |    |   (  <_> )  \___|    < | \_\ (  <_> > <  <
%   Firmware   |____|_  /\____/ \___  >__|_ \|___  /\____/__/\_ \
%                     \/            \/     \/    \/            \/
%                                   - M A N U A L -
%
% $Id$
%
% Copyright (C) 2006 The Rockbox Manual Team
%
% All files in this archive are subject to the GNU Free Documentation License
% See the file COPYING-MANUAL in the source tree docs/ directory for full license
% agreement.
%
% Contributors are listed in the file docs/CREDITS-MANUAL
%
%%%%%%%%%%%%%%%%%%%%%%%%%%%%%%%%%%%%%%%%%%%%%%%%%%%%%%%%%%%%%%%%%%%%%%%%%%%%%

\documentclass[a4paper,11pt,hyphens]{scrreprt}
\usepackage[utf8]{inputenx}
\usepackage[T1]{fontenc}
% load ix-utf8enc.dfu to get additional characters from the inputenx package.
\makeatletter\input{ix-utf8enc.dfu}\makeatother
% This manual used to use Palatino as font. This caused issues with small caps
% (textsc), so do not use that font as default one anymore.
% Use Latin Modern instead which improves searchability of the PDF output.
\usepackage{lmodern}

\usepackage{tabularx}
\usepackage{multirow}
\usepackage{multicol}

\usepackage{scrhack}
\usepackage{float}
\floatstyle{ruled}

\usepackage{hyperref}
\usepackage{enumitem}

\usepackage{optional}

\usepackage[table]{xcolor}

\input{platform/\platform.tex}
\input{features.tex}

\newcommand{\playername}{\playerman\ \playerlongtype}

\hypersetup{% add some values to the pdf properties
  colorlinks = true,
  pdfstartview = FitV,
  linkcolor = blue,
  citecolor = blue,
  urlcolor = blue,
  pdftitle = {Rockbox user manual},
  pdfauthor = {The Rockbox Team},
  pdfsubject = {Rockbox user manual for \playername}
}

\newcommand{\fname}[1]{\texttt{#1}}
\newcommand{\tabeltc}[1]{{\centering #1 \par}}
\newcommand{\tabelth}[1]{{\centering \textbf{\textit{#1}} \par}}

\newcommand{\fixme}[1]{\textbf{\textcolor{red}{#1}}}

% Colors used to typeset table headers and alternating table rows
\colorlet{tblhdrbgcolor}{blue!30} % Background color for headers
\colorlet{tbloddrowbgcolor}{blue!10} % Background color for odd rows (headers not included)
\colorlet{tblevenrowbgcolor}{white} % Background color for even rows (headers not included)

\usepackage{fancyhdr}
\usepackage{graphicx}
\usepackage{verbatim}
\usepackage{makeidx}
\usepackage{fancyvrb}
\usepackage{color}
\ifpdfoutput{\usepackage{booktabs}}%
  {\newcommand{%
    \toprule}{}\newcommand{\midrule}{\hline}\newcommand{\bottomrule}{}%
  }
\usepackage{longtable}
\usepackage{url}
\urlstyle{sf}
\Urlmuskip = 0mu plus 1mu
\usepackage{marvosym}
\usepackage{rotating}

% pdf output: try pdf first, then png and jpg as file format
% html output: try png first, then jpg. Ignore pdf files
% this only applies if no file extension is given!
\ifpdfoutput{\DeclareGraphicsExtensions{.pdf,.png,.jpg}}%
    {\DeclareGraphicsExtensions{.png,.jpg}}

% fancy header style adjustments
\fancyhead{}
\fancyfoot{}
\fancyhead[L]{{\nouppercase{\textsc{\leftmark}}}}
\fancyhead[R]{\iffloatpage{}{\thepage}}
\fancyfoot[L]{\textsc{The Rockbox manual} {\scriptsize(version \buildversion)}}
\fancyfoot[R]{\textsc{\playerman{} \playertype}}
\fancypagestyle{plain}{}

\renewcommand{\headrulewidth}{\iffloatpage{0pt}{0.4pt}}
\renewcommand{\footrulewidth}{\iffloatpage{0pt}{0.4pt}}
\setlength{\headheight}{18.5pt}
\newcounter{example}[chapter]

\ifpdfoutput{%
  \renewcommand{\toprule}{\specialrule{\heavyrulewidth}{\abovetopsep}{0pt}}
  \renewcommand{\midrule}{\specialrule{\lightrulewidth}{0pt}{\belowrulesep}}
  }{}
\newcommand{\tblhdrstrut}{\rule[-1.3ex]{0mm}{4.0ex}}


%% \newenvironment{example}
%%     {\stepcounter{example}\paragraph{Example \theexample:}}
%%     {\hfill$\Box$
    
%%     \bigskip
%%     \noindent}

% found on the internet, posting by Donald Arseneau
% I may as well include my robust expandable definions, which can be
% used in \edef or \write where the \def would not be executed:
%
% \if\blank --- checks if parameter is blank (Spaces count as blank)
% \if\given --- checks if parameter is not blank: like \if\blank{#1}\else
% \if\nil --- checks if parameter is null (spaces are NOT null)
% use \if\given{ } ... \else ... \fi etc.
%
{\catcode`\!=8 % funny catcode so ! will be a delimiter
\catcode`\Q=3 % funny catcode so Q will be a delimiter
\long\gdef\given#1{88\fi\Ifbl@nk#1QQQ\empty!}
\long\gdef\blank#1{88\fi\Ifbl@nk#1QQ..!}% if null or spaces
\long\gdef\nil#1{\IfN@Ught#1* {#1}!}% if null
\long\gdef\IfN@Ught#1 #2!{\blank{#2}}
\long\gdef\Ifbl@nk#1#2Q#3!{\ifx#3}% same as above
} 

% environment for setting the changelog.
\newenvironment{changelog}%
    {\renewcommand{\labelitemi}{$\star$}\setitemize{noitemsep,topsep=0pt}%
    \begin{itemize}}%
    {\end{itemize}}


\newcommand{\dapdisplaysize}{\dapdisplaywidth$\times$\dapdisplayheight$\times$\dapdisplaydepth}
\newcommand{\genericimg}{\dapdisplaywidth x\dapdisplayheight x\dapdisplaydepth}

% add screenshot image.
% Usage: \screenshot{filename}{caption}{label}
% By using the 'H' (HERE) placement, the screenshots are placed where
% we want them.
% Note: use this only for screenshots!
% Note: leave caption empty to supress it.
% Resulting file names in the source should consist of up to 3 parts
% "filename-\genericimg-[\specimg|\seriesimg]"  (the third is optional)
% filename is specified by the use of this macro in the tex file
% \genericimg will be generated using the display's resolution (see above)
% \specimg or \seriesimg are possible options set in the platform files
% Files will be used in the following order: 1.filename contains \specimg part
% 2. filename contains \seriesimg part, 3. filename without the third part

% set seriesimg if it isn't defined in the platform file to not break manuals
\ifdefined\seriesimg
\else
    \newcommand{\seriesimg}{\specimg}
\fi

\newcommand{\screenshot}[3]{
  \begin{figure}[H]
    \begin{center}
      \IfFileExists{#1-\genericimg-\specimg.png}
        {\includegraphics[width=\screenshotsize]{#1-\genericimg-\specimg.png}
         \typeout{Note: device specific image used}}
        {\IfFileExists{#1-\genericimg-\seriesimg.png}
          {\includegraphics[width=\screenshotsize]{#1-\genericimg-\seriesimg.png}
           \typeout{Note: series specific image used}}
          {\IfFileExists{#1-\genericimg.png}
            {\includegraphics[width=\screenshotsize]{#1-\genericimg.png}}
            {\IfFileExists{#1}
              {\includegraphics[width=\screenshotsize]{#1}
               \typeout{Warning: deprecated plain image name used}}%
              {\typeout{Missing image: #1 (\genericimg) (\specimg)}%
               \color{red}{\textbf{WARNING!} Image not found}%
            }
          }
        }
      }
      \if\blank{#3}\else{\label{#3}}\fi\if\blank{#2}\else{%
        \caption{#2}}\fi
    \end{center}
  \end{figure}
}

% command to display a note.
% Usage: \note{text of your note}
% Note: do NOT use \textbf or similar to emphasize text, use \emph!
\ifpdfoutput{
\newcommand{\note}[1]{
  \ifinner\else\vspace{1ex}\par\noindent\fi
  \textbf{Note:}\ %
  \ifinner#1\else\marginpar{\raisebox{-6pt}{\Huge\Writinghand}}#1\par\vspace{1ex}\fi%
}}
{\newcommand{\note}[1]{\ifinner\else\par\noindent\fi\textbf{Note:{} }#1\par}}

% command to display a warning.
% Usage: \warn{text of your warning}
% Note: do NOT use \textbf or similar to emphasize text!
\ifpdfoutput{
\newcommand{\warn}[1]{
  \ifinner\else\par\noindent\fi
  \textbf{Warning:\ }%
  \ifinner#1\else\marginpar{\raisebox{-6pt}{\Huge\Stopsign}}#1\par\fi%
}}
{\newcommand{\warn}[1]{\ifinner\else\par\noindent\fi\textbf{Warning:{} }#1}}

% command to mark a text block as intended especially for blind people
% Usage: \blind{text}
\ifpdfoutput{
\newcommand{\blind}[1]{\mbox{}\marginpar{\raisebox{-1ex}{\Huge{\ForwardToEnd}}}#1}
}
{\newcommand{\blind}[1]{\ifinner\else\par\noindent\fi#1}}

% make table floats use "H" (as for screenshots) as default positioning
\makeatletter\renewcommand{\fps@table}{H}\makeatother
% change defaults for floats on a page as we have a lot of small images
\setcounter{topnumber}{3}    % default: 2
\setcounter{bottomnumber}{2} % default: 1
\setcounter{totalnumber}{5}  % default: 3


% Environment for typesetting tables in a consistent way. The header has
% a darker background; rows are white and light gray (alternating). Top,
% middle and bottom rules are automatically set.
%
% Params:
%    #1 -- table width
%    #2 -- column specification (as used in the package tabularx)
%    #3 -- contents of the header row. The number of items must
%          match the number of specs in #2
%    #4 -- caption (optional). If used then this must be inside a floating
%          environment, e.g. table
%    #5 -- label (optional)
%
% Example:
% \begin{rbtabular}{0.9\textwidth}{lX}{Col1 & Col2}{}{}
% A1 & A2\\
% B1 & B2\\
% C1 & C2\\
% \end{rbtabular}
%    
\newenvironment{rbtabular}[5]{%
  \rowcolors{2}{tbloddrowbgcolor}{tblevenrowbgcolor}
  \expandafter\let\expandafter\SavedEndTab\csname endtabular*\endcsname
  \expandafter\renewcommand\expandafter*\csname endtabular*\endcsname{%
    \bottomrule
    \SavedEndTab%
    \if\given{#4}\caption{#4}\fi%
    \if\given{#5}\label{#5}\fi%
    \endcenter%
  }
  \center
  \tabularx{#1}{#2}\toprule\rowcolor{tblhdrbgcolor}
  \tblhdrstrut#3\\\midrule
}{%
  \endtabularx
}

\newcommand{\tabnlindent}{\newline\mbox{ }\mbox{ }}



% command to set the default table heading for button lists
\newcommand{\taghead}{\tblhdrstrut\textbf{Tag} & \textbf{Description} \\\midrule}

% environment intended to be used with tag maps (for wps)
% usage: \begin{tagmap}{caption}{label} Tag & Description \\ \end{tagmap}
% Note: this automatically sets the table lines.
% Note: you *need* to terminate the last line with a linebreak \\
% Cheers for the usenet helping me building this up :)
\newenvironment{tagmap}{%
\rowcolors{2}{tbloddrowbgcolor}{tblevenrowbgcolor}
  \expandafter\let\expandafter\SavedEndTab\csname endtabular*\endcsname
  \expandafter\renewcommand\expandafter*\csname endtabular*\endcsname{%
    \bottomrule
    \SavedEndTab%
    \endcenter%
  }
  \center
\tabularx{\textwidth}{Xp{.7\textwidth}}\toprule\rowcolor{tblhdrbgcolor} % here is the table width defined
  \taghead
}{%
  \endtabularx
}

% When creating HTML, use the soul package.
% This produces much nicer HTML code (textsc results in each character being
% put in a separate <span>).
\ifpdfoutput{\newcommand{\caps}[1]{\textsc{#1}}}{\usepackage{soul}}
\newcommand{\setting}[1]{\caps{#1}}

\newcommand{\config}[1]{\texttt{#1}}

% set link to download server
% Usage: \download{bootloader/bootloader-ipodnano.ipod}
%        gets expanded to 
%        "https://download.rockbox.org/bootloader/bootloader-ipodnano.ipod"
\newcommand{\download}[1]{\url{https://download.rockbox.org/#1}}

% set link to the wiki.
% Usage: \wikilink{WebHome}
%        with "WebHome" being the wiki page name
\newcommand{\wikibaseurl}{https://www.rockbox.org/wiki/}
\ifpdfoutput{\newcommand{\wikiicon}{\Pointinghand}}
    {\newcommand{\wikiicon}{}}
\newcommand{\wikilink}[1]{\wikiicon{}\href{\wikibaseurl#1}{#1}}
%\newcommand{\wikilink}[1]{\url{https://www.rockbox.org/wiki/#1}}

% define environment "code" based on fancyvrb.
% use it to set code the user should type / see on his screen.
% Note: the first 4 characters of each line will be stripped,
%       requiring everything to be indendet by exactly _4_ spaces!
%       This is intended to make the LaTeX sources more readable.
% Note: when using the code environment you need to use optv instead of opt!
\DefineVerbatimEnvironment{code}{Verbatim}%
  {framerule=0.4pt, framesep=1ex,frame=lines,%numbers=left,stepnumber=5,%
   gobble=4,fontsize=\footnotesize,xleftmargin=10pt,xrightmargin=10pt,%
   label=\textnormal{\textsc{Code}},%
   commandchars=\\\{\}%
   }

% define environment "example" based on fancyvrb.
% use it to set example code the user should type / see on his screen.
% Note: the first 4 characters of each line will be stripped,
%       requiring everything to be indendet by exactly _4_ spaces!
%       This is intended to make the LaTeX sources more readable.
% Note: when using the example environment you need to use optv instead of opt!
\DefineVerbatimEnvironment{example}{Verbatim}%
  {commentchar=!,framerule=0.4pt, framesep=1ex,frame=lines,%
   gobble=4,fontsize=\footnotesize,xleftmargin=10pt,xrightmargin=10pt,%
   label=\textnormal{\textsc{Example}},%
   commandchars=\\\{\}%
   }

% Use the nopt command to exclude a certain defined feature from a section
% Example:
% \nopt{ondio}{This text will be excluded for ondios}
\makeatletter
\newcommand*\nopt[1]{\if\Opl@notlisted{#1}\expandafter\@firstofone
   \else \expandafter\@gobble \fi}
\makeatother

% Use the reference to handle get a flexible reference command
% usage \reference{ref_section}
\ifpdfoutput{%
  \newcommand{\reference}[1]{%
    section~\ref{#1} (page~\pageref{#1})%
}}
{\newcommand{\reference}[1]{section~\ref{#1}}}

% special HTML output adjustments
\ifpdfoutput{}{\setlength{\parindent}{0mm}}
\ifpdfoutput{}{\renewcommand{\Forward}[1]{$\triangleright${}#1}}

\newcommand{\btnfnt}[1]{\textbf{#1}}
%\hfuzz=2pt

% generate index
\usepackage{makeidx}
\makeindex


% command to set the default table heading for button lists
\newcommand{\btnhead}{\textbf{Key} \opt{HAVEREMOTEKEYMAP}{%
  & \textbf{Remote Key}} & \textbf{Action} \\\midrule}
% environment intended to be used with button maps
% usage: \begin{btnmap} Button & ButtonAction \\ \end{btnmap}
% Note: this automatically sets the table lines.
% Note: you *need* to terminate the last line with a linebreak \\
% Cheers for the usenet helping me building this up :)

% tabularx is set to be either two or three columns wide depending on whether
% HAVEREMOTEKEYMAP is defined in the platform file for the target in question.
% If it is, then every button table has three columns of the form
% Main Unit Key  &  Remote Key  &  Description \\

\newenvironment{btnmap}{%
  \rowcolors{2}{tbloddrowbgcolor}{tblevenrowbgcolor}
  \expandafter\let\expandafter\SavedEndTab\csname endtabular*\endcsname
  \expandafter\renewcommand\expandafter*\csname endtabular*\endcsname{%
    \bottomrule
    \SavedEndTab%
    \endcenter\vspace{2ex}%
  }
  \vspace{2ex}\center
  \opt{HAVEREMOTEKEYMAP}{
    % here is the table width defined for 3 columns
    \tabularx{.95\textwidth}{>{\raggedright\arraybackslash}p{.2\textwidth}>{\raggedright\arraybackslash}p{.2\textwidth}X}\toprule\rowcolor{tblhdrbgcolor}
  }
  \nopt{HAVEREMOTEKEYMAP}{
    % here is the table width defined for 2 columns
    \tabularx{.75\textwidth}{>{\raggedright\arraybackslash}p{.22\textwidth}X}\toprule\rowcolor{tblhdrbgcolor}
  }
  \tblhdrstrut\btnhead
}{%
  \endtabularx
}
