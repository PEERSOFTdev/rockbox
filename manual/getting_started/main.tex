% $Id$ %
\chapter{Introduction}
\section{Welcome}
This is the manual for Rockbox. Rockbox is an open source firmware replacement
for a growing number of digital audio players. Rockbox aims to be considerably
more functional and efficient than your device's stock firmware while remaining
easy to use and customisable. Rockbox is written by users, for users. Not only
is it free to use, it is also released under the GNU General Public License
(GPL), which means that it will always remain free both to use and to change.

Rockbox has been in development since 2001, and receives new features, tweaks
and fixes each day to provide you with the best possible experience on your
digital audio player. A major goal of Rockbox is to be simple and easy to use,
yet remain very customisable and configurable. We believe that you should never
need to go through a series of menus for an action you perform frequently. We
also believe that you should be able to configure almost anything about Rockbox
you could want, pertaining to functionality. Another top priority of Rockbox is
audio playback quality -- Rockbox, for most models, includes a wider range of
sound settings than the device's original firmware. A lot of work has been put
into making Rockbox sound the best it can, and improvements are constantly being
made. All models have access to a large number of plugins, including many games,
applications, and graphical ``demos''. You can load different configurations
quickly for different purposes (e.g. a large font for in your car, different
sound settings for at home). Rockbox features a very wide range of languages, and
all supported models also have the ability to talk to you -- menus can be voiced
and filenames spelled out or spoken.

\section{Getting more help}
This manual is intended to be a comprehensive introduction to the Rockbox
firmware. There is, however, more help available. The Rockbox website at
\url{https://www.rockbox.org/} contains very extensive documentation and guides
written by members of the Rockbox community and this should be your first port
of call when looking for further help.

If you cannot find the information you are searching for on the Rockbox
website there are a number of support channels you should have a look at.
You can try the Rockbox forums located at \url{https://forums.rockbox.org/}.
The mailing lists are another option, and can be found at
\url{https://www.rockbox.org/mail/}. From that page you can subscribe to the
lists and browse the archives. To search the list archives simply use
the search field that is located on the left side of the website.
Furthermore,  you can ask on IRC. The main channel for Rockbox is
\texttt{\#rockbox} on \url{irc://irc.freenode.net}. Many helpful developers
and users are usually around. Just join and ask your question (don't ask to
ask!) -- if someone knows the answer you'll
usually get an answer pretty quickly. More information including IRC logs
can be found at \url{https://www.rockbox.org/irc/}. We also have a web client
so that you can join the Rockbox IRC channel without needing
to install additional software onto your computer.

If you think you have found a bug please make sure it actually is a bug and is
still present in the most recent version of Rockbox. You should try to
confirm that by using the above mentioned support channels first. After that
you can submit that issue to our tracker. Refer to \reference{sec:feedback}
for details on how to use the tracker.


\section{Naming conventions and marks}
We have some conventions (especially for naming) that are intended to be
consistent throughout this manual.

Manufacturer and product names are formatted in accordance with the standard
rules of English grammar, e.g. ``\playerman{} playback is currently
unsupported''. Manufacturer and model names are proper nouns, and
thus are written beginning with a capital letter.

% write a bit more about names etc. here.
\ifpdfoutput{
This manual has some parts that are marked with icons on the margin to help
you finding important parts or parts you could skip. The following icons
are used:
\\
\note{This indicates a note. A note starts always with the text ``Note''.
  In order to make finding notes easier each one is accompanied by
  an icon in the margin as here. Notes are used to mark useful information
  that may help you to get the most out of Rockbox.
\\
}
\warn{This is a warning. In contrast to notes mentioned above, a warning
  should be taken more seriously. Whereas ignoring notes will not cause any
  serious damage, ignoring warnings \emph{could} cause serious damage to 
  your \dap{}. You really should read the warnings, especially if you are
  new to Rockbox.
\\
}
\blind{This icon marks a section that is intended especially for the blind
  and visually impaired. As they cannot
  read the manual in the same way sighted people do we have added some
  additional descriptions. If you are not blind or visually impaired you
  can probably completely skip these blocks. To make this easier, there is an
  icon shown in the margin on the right.
\\
}
}{}% end ifpdfoutput

Links to the wiki are abbreviated by the name of the wiki page. Those names
are still linked so you can simply follow them like any other link in this
manual. If you want to access a wiki page manually go to
\wikiicon{} \href{\wikibaseurl}{\wikibaseurl}
and type the page name in the ``Go'' box at the top of the page.
\ifpdfoutput{Links to wiki pages are also indicated by the symbol \wikiicon{}
in front of the page name.}{}

\input{getting_started/installation.tex}
